\subsection{Indeling navolgende rondes}

\begin{enumerate}
\item Indien een toernooi drie ronden in totaal kent, wordt de tweede ronde als tussenronde aangemerkt. In geval van vier ronden worden de ronden twee en drie als tussenronde aangemerkt.
\item Voor de indeling van deze rondes worden alle teams gerangschikt. Dit gebeurt primair op basis van het aantal behaalde overwinningen en secundair op basis van sprekerspunten.
\item Het hoogst gerangschikte team wordt ingedeeld om te debatteren tegen het op één na hoogst gerangschikte team. Zij worden ingedeeld als voor- of tegenstanders afhankelijk van welke rol zij het minst hebben vervuld. Indien de teams van dezelfde school zijn, het eerder tegen elkaar hebben opgenomen, of een team een positie drie keer zou moeten vervullen, wordt het hoogst gerangschikte team ingedeeld om te debatteren tegen het op twee na hoogst gerangschikte team. 
\item Indien twee teams het tegen elkaar kunnen opnemen, worden zij uit de lijst van in te delen teams gehaald. De in lid 34.2 beschreven procedure herhaalt zich totdat alle teams zijn ingeschreven.
\item Indien de procedure als beschreven in de leden 34.2 en 34.3 niet tot een indeling voor alle teams leidt, wordt van de vereisten afgeweken. De uiteindelijke indeling voldoet zoveel mogelijk aan de vereisten.
\end{enumerate}
