\subsection{Algemene bepaling}

\begin{enumerate}
\item Teams van dezelfde school debatteren niet tegen elkaar.
\item Teams debatteren in beginsel niet meer dan één keer tegen elkaar. 
\item Het verschil tussen het aantal keer dat een team voor of tegen is, bedraagt niet meer dan een.
\item De indeling is een verantwoordelijkheid van de wedstrijdleiding. Zij kan hierbij gebruik maken van software.
\item Van elke regel voor indelingen, waaronder ook de leden 32.1, 32.2 en 32.3 kan worden afgeweken. 
\item De wedstrijdleiding deelt ook de juryleden in, waarbij zij in beginsel voor elke debater een zo goed mogelijke jurering probeert te verzorgen. Dit houdt in dat de wedstrijdleiding onder andere in acht neemt dat juryleden niet scholen jureren met wie zij een persoonlijke band hebben dat er indelingen ontstaan die anderszins (de schijn van) partijdigheid kunnen opleveren. De organisatie maakt het mogelijk voor deelnemers en juryleden om dit soort clashes tijdig kenbaar te maken. 
\end{enumerate}
