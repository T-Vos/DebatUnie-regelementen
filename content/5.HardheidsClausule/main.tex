\hiddenchapter{{\large{A}lgemene Hardheidsclausulen}}
\ChapterTitle[\thechapter]{Algemene Hardheidsclausule}
%%% See Preamble for an explanation of both of these %%%

\begin{enumerate}
\item Stichting DebatUnie is bevoegd op elk moment op elke wijze van dit reglement af te wijken, indien strikte toepassing van dit reglement tot evident onredelijke uitkomsten leidt.
\item In gevallen waarin dit reglement niet voorziet, bepaalt het bestuur van Stichting DebatUnie. Op toernooien komt deze bevoegdheid toe aan de wedstrijdleiding, in alle andere gevallen aan het bestuur van de stichting. 
\item Stichting DebatUnie kan onder meer uitslagen van individuele debatten of toernooien nietig verklaren, vervangende uitkomsten vaststellen, specifieke teams uitnodigen voor de finaledag en teams of deelnemers verwijderen.
\item Indien een maatregel meerdere deelnemers treft, probeert Stichting DebatUnie de nadelen van de maatregel altijd zo gelijk mogelijk over de getroffen deelnemers te verspreiden. Nadelen voor deelnemers jegens wie de maatregel niet gericht is, worden zo veel mogelijk beperkt.
\item Afwijking van of invulling van leemtes in het reglement wordt zo snel mogelijk bekend gemaakt en gemotiveerd, tenzij het belang van vertrouwelijkheid zich daartegen verzet.
\item Indien andere personen of organisaties met handhaving zijn belast, gelden de bevoegdheden van de leden 45.1 en 45.2 ook voor hen. Zij kunnen slechts afwijken van de artikelen met uitvoering waarvan zij zijn belast. De leden 45.3, 45.4 en 45.5 zijn van overeenkomstige toepassing.
\item Bij strijdigheid van het reglement kan Stichting DebatUnie maatregelen nemen, waaronder het nietig verklaren van de uitslag van een debat danwel het vaststellen van een vervangende uitkomst.
\end{enumerate}
