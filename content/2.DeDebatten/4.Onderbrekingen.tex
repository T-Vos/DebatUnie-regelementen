\hiddensection{{\large{O}}nderbreking }
\SectionTitle{\thesection}{Onderbreking }

\begin{enumerate}
\item De jury onderbreekt het debat slechts wanneer dat strikt noodzakelijk is. Zij stelt de wedstrijdleiding zo spoedig als redelijk mogelijk van de onderbreking op de hoogte.
\item Bij onderbreking van het debat kijken jury en wedstrijdleiding of een spoedige hervatting van het debat mogelijk is.
\item Indien een spoedige hervatting van het debat niet mogelijk is en één van de teams in staat is verder te debatteren, wordt dit team als winnaar aangemerkt.
\item Indien een team veroorzaakt dat het andere team niet verder kan debatteren, vindt lid 11.3 geen toepassing. Het team dat niet in staat is verder te debatteren wordt dan als winnaar aangewezen.
\item Indien een spoedige hervatting van het debat niet mogelijk is en geen van de teams in staat is verder te debatteren, wordt de winnaar van het debat door loting bepaald. 
\item De sprekerspunten van alle sprekers worden bepaald door het gemiddelde te nemen van hun sprekerspunten in de overige rondes.
\item Indien een groot aantal van de debatten in één ronde wordt onderbroken en spoedige hervatting niet mogelijk is, kan de wedstrijdleiding bepalen dat de ronde komt te vervallen. Er worden dan geen scores toegekend en in de volgende ronde debatteren de teams of een deel van de teams weer tegen elkaar.
\item Indien een team gedurende aan een toernooi aan minder dan twee debatten heeft deelgenomen, wordt dit team niet opgenomen in de uitslag.
\end{enumerate}