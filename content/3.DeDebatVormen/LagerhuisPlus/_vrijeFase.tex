\subsection{De vrije fase}

\begin{enumerate}
\item De juryvoorzitter wijst een jurylid aan dat tijdens de vrije fase als gespreksleider optreedt. De voorzitter kan ook zelf als gespreksleider optreden. 
\item Deelnemers aan het debat maken kenbaar dat zij het woord willen voeren door op te staan. In geval van fysieke beperking of om andere redenen kunnen zij op andere manieren kenbaar maken het woord te willen voeren. Zij geven voor het begin van het debat aan hoe ze kenbaar maken dat ze het woord willen voeren.
\item De gespreksleider wijst aan wie, van de mensen het woord willen voeren, het woord krijgt. Hierbij beoogt hij primair een eerlijke verdeling van spreekbeurten over de teams en secundair een eerlijke verdeling van spreekbeurten over de sprekers.
\item Een spreekbeurt in de vrije fase duurt maximaal 30 seconden. De uitlooptijd als bedoeld in lid 10.3 is niet van toepassing. Na 20 seconden maant de gespreksleider de spreker tot afronden. Na 30 seconden onderbreekt de gespreksleider de spreker en neemt de jury wat daarna wordt gezegd niet langer mee in zijn beoordeling.
\item Spreken zonder dat door de gespreksleider het woord is verleend is een overtreding van de regels. De jury kan een team of spreker hierdoor als minder overtuigend beoordelen. De gespreksleider is in uitzonderlijke gevallen bevoegd een spreker te verbieden het woord te voeren voor de duur van de gehele vrije fase. Hij doet dit niet lichtzinnig.
\item In aanvulling op artikel 13 beoordeelt de jury ook de prestaties van teams gedurende de vrije fase met sprekerspunten. Zij doet dit door eerst de kwaliteit van de vrije fase in het algemeen met een puntenaantal te beoordelen en vervolgens deze punten over de teams te verdelen aan de hand van hun prestatie.
\end{enumerate}
